\documentclass[11pt,letterpaper,draft]{report}
\usepackage[left=3.5cm,right=3.5cm,top=2.5cm,bottom=2.5cm]{geometry}
\usepackage[english]{isodate}

\usepackage{amsmath,amsfonts,amsthm,bm,stmaryrd,amssymb}
\usepackage{enumitem,mathtools,multicol}
\setenumerate[1]{label=(\alph*)}
\setenumerate[0]{label=\roman*.}
\allowdisplaybreaks

\setcounter{secnumdepth}{3}
\setcounter{tocdepth}{3}

\newtheorem{defn}{Definición}
\newtheorem{example}{Ejemplo}[section]
\newtheorem{exe}{Ejercicio}
\newtheorem{lemma}{Lema}
\newtheorem{rem}{Recordatorio}
\newtheorem*{sol}{Solución}

\newcommand\R{\mathbb R}
\newcommand\C{\mathbb C}
\newcommand\Z{\mathbb Z}
\newcommand\N{\mathbb N}
\newcommand\norm[1]{\left\|#1\right\|}
\newcommand\<{\langle}
\renewcommand\>{\rangle}
\renewcommand\phi\varphi
\let\cal\mathcal
\let\ol\overline
\DeclareMathOperator{\erf}{erf}
\DeclareMathOperator{\sgn}{sgn}
\DeclareMathOperator{\csch}{csch}
\title{Notas}
\author{Jorge Alfredo Álvarez Contreras}

\renewcommand\contentsname{Contenido}

\begin{document}
\maketitle

\tableofcontents

\chapter{Series de Fourier}

\begin{defn}
  Dos funciones son ortogonales en $[a,b]$ si $\int_a^bfg=0$.
\end{defn}
\begin{example}
  \begin{enumerate}
    \item si $f(x)=2x^2$ y $g(x)=4x^3$, ambas definidas en
    $[-1,1]$, entonces
    \begin{align*}
      \int_{-1}^12x^2\cdot 4x^3\,dx
      &= 8\int_{-1}^1x^5\,dx \\
      &= \frac{8}{6}x^6|_{-1}^1 \\
      &= 0
    \end{align*}
    \item Sea $f(x)=4x-1$ y $g(x)=x^2-2x$ definidas en $[0,2]$.
    Entonces
    \begin{align*}
      \int_0^2(4x-1)(x^2-2x)\,dx
      &= \int_0^2(4x^3-9x^2+2x)\,dx \\
      &= x^4-3x^3+x^2 |_0^2 \\
      &= -4 \\
      &\neq 0.
    \end{align*}
  \end{enumerate}
\end{example}

\begin{defn}
  Un conjunto infinito de funciones $\{\phi_n\}_{n=0}^\infty$ se
  dice que es ortogonal en $[a,b]$ si
  \begin{align*}
    \int_a^b\phi_n\phi_m &= 0 & \text{ para } n\neq m.
  \end{align*}
  Si $n=m$, entonces
  \[
    \<\phi_n,\phi_m\> = \int_a^b \phi_n^2
  \]
  es la norma cuadrada de $\phi_n$ que se denotará como
  $\norm{\phi_n}$.
\end{defn}

\begin{example}
  \begin{enumerate}
    \item   
    El conjunto $\{1,\cos x,\cos 2x,\cos 3x,\dots\}$ es ortogonal
    en $[-\pi,\pi]$, ya que, si $\phi_n(x)=\cos nx$,
    $n=1,2,3,\dots$ y $\phi_0(x)=1$, entonces
    \begin{align*}
      \int_{-\pi}^\pi \phi_n\phi_m
      &= \int_{-\pi}^\pi \cos nx \cos mx \,dx \\
      &= \frac{1}{2}
        \int_{-\pi}^\pi [\cos((m+n)x) + \cos((m-n)x)] \,dx \\
      &= \frac{1}{2(m+n)}\sin((m+n)x) + \frac{1}{2(m-n)}\sin((m-n)x)
      |_{-\pi}^\pi \\
      &= 0.
    \end{align*}
    para $n,m\geq 0$, $n\neq m$.

    \item Determine si el conjunto $\{\sin x,\sin 2x,\sin
    3x,\dots\}$ es ortogonal en $[0,2\pi]$.
    Tenemos
    \begin{align*}
      \int_0^\pi \sin nx\sin mx \,dx
      &= \frac{1}{2}\int_0^\pi[\cos((m-n)x)-\cos((m+n)x)] \,dx \\
      &= \frac{1}{2(m-n)}\sin((m-n)x)
        - \frac{1}{2(m+n)}\sin((m+n)x) |_0^\pi \\
      &= 0.
    \end{align*}
  \end{enumerate}
\end{example}

\chapter{Métodos de solución. (Mie 10 nov 2021)}

Los métodos más usuales para resolver los P.F. son
\begin{enumerate}
  \item Separación de variables
  \item Por transformadas integrales
  \begin{itemize}
    \item   Laplace
    \item Fourier   
  \end{itemize}
  \item Funciones de Green
  \item Funciones variacionales
  \item Métodos numéricos
\end{enumerate}

\section{Método por separación de variables}

\subsection{Problemas de calor}

\subsubsection{Problemas con temperatura cero}
Considérese el problema de determinar la distribución de
temperatura en una varilla delgada lateralmente aislada de
longitudo $L$ en donde la temperatura en los extremos es cero y
con temperatur ainicial dada por $f(x)$ a lo largo de la varilla.

\paragraph{Modelo}
Sea $u=u(x,t)$ la función que describe la temperatura sobre la
varilla. Entonces la ecuación diferencial es
\[
  \frac{\partial u}{\partial t}
  = k
  \frac{\partial^2 u}{\partial x^2},
  \hspace{10mm} 0<x<L, t>0
.\]
\begin{align*}
  C.F. && u(0,t) &= 0 & u(L,t) &= 0 &t&>0 \\
  C.I. && u(x,0) &= f(x) & &&& 0<x<L
\end{align*}

\paragraph{Solución}
El método de separación de variables supone como solución a
\[
  u(x,t) = X(x)T(t)
.\]
Sustituyendo en la ecuación diferencial, tenemos $XT'=kX''T$.
Dividiendo entre $ku=kXT$, tenemos
\[
  \frac{XT'}{kXT} = k \frac{X''T}{kXT}
.\]
Es decir,
\[
  \frac{T'}{kT} = \frac{X''}{X}
.\]
Como el lado izquierdo depende solo de $t$ y el lado derecho
depende solo de $X$, la igualdad anterior solo es posible cuando
ambos lados son iguales a una constante. La denotaremos como
$-\lambda$ por conveniencia.
Así,
\[
  \frac{T'}{kT} = \frac{X''}{X} = -\lambda
.\]
A $\lambda$ la llamaremos la constante de separación.
Por lo tanto, obtenemos dos EDO
\begin{align*}
  X'' + \lambda X &= 0
  &
  T' + kT &= 0
\end{align*}
Ahora, aplicando las condiciones de frontera a $u(x,t)$:
\begin{align*}
  \text{Si } u(0,t)&=0
  & &\implies &
  X(0)T(t)&=0
  & &\text{ ssi } &
  X(0)&=0 \\
  \text{Si } u(L,t)&=0
  & &\implies &
  X(L)T(t)&=0
  & &\text{ ssi } &
  X(L)&=0
\end{align*}
Así, obtenemos un problema de Strum-Liouville cuya solución
depende de $\lambda$.

\begin{itemize}
  \item \emph{Caso $\lambda=0$.}
  Entonces la ecuación diferencial $X''=0$ tiene solución
  \[
    X(x)=c_1+c_2x
  .\]
  \begin{itemize}
    \item
    Como $X(0)=0$, entonces $c_1+0=0$, por lo cual $c_1=0$.
    \item
    Como $X(L)=0$, entonces $c_2L=0$, por lo cual $c_2=0$.
  \end{itemize}
  Concluimos que $X(x)=0$ para todo $x$: la solución es trivial.

  \item \emph{Caso $\lambda<0$.}
  Entonces la ecuación diferencial tiene solución
  \[
    X(x) = c_1e^{\alpha x} + c_2 e^{-\alpha x}
  ,\]
  donde $\alpha=\sqrt{-\lambda}$.
  \begin{itemize}
    \item
    Como $X(0)=0$, entonces $c_1+c_2=0$, por lo cual $c_1=-c_2$.
    \item
    Como $X(L)=0$, entonces $c_1e^{\alpha L}+c_2e^{-\alpha L}=0$,
    por lo cual $c_1(e^{\alpha L}-e^{-\alpha L})=0$. Esto implica
    que $c_1=0$.
  \end{itemize}
  Concluimos que $X(x)=0$ para todo $x$: la solución es trivial.

  \item \emph{Caso $\lambda>0$.}
  Entonces la ecuación diferencial tiene solución
  \[
    X(x)= c_1\cos(\beta x)+c_2\sin(\beta x)
  ,\]
  donde $\beta=\sqrt{\lambda}$.
  \begin{itemize}
    \item Como $X(0)=0$, entonces $c_1=0$.
    \item Como $X(L)=0$, entonces $c_2\sin(\beta L)=0$.
  \end{itemize}
  Considerando $c_1=0$, concluimos que $\beta L=\pi n$ con $n$
  entero positivo. Así,
  \[
    \beta= \frac{n\pi}{L}, \hspace{10mm} n=1,2,3,\dots
  .\]
  Concluimos que
  $X(x)=c_2\sin(\tfrac{n\pi}{L}x)$ para $n=1,2,3,\dots$.
\end{itemize}

Ahora, sustituyendo $\lambda=(\tfrac{n\pi}{L})^2$ en la ecuación
para $T(t)$, tenemos
\[
  T' + k (\tfrac{n\pi}{L})^2 T = 0
.\]
La cual tiene solución $T(t)=c_3\exp(-k(\tfrac{n\pi}{L})^2t)$.
Por lo tanto, tenemos una familia de soluciones
\begin{align*}
  u_n(x,t)
  &= X(x)T(t) \\
  &= c_2\sin(\tfrac{n\pi}{L}x)c_3\exp(-k(\tfrac{n\pi}{L})^2t) \\
  &= b\sin(\tfrac{n\pi}{L}x)\exp(-k(\tfrac{n\pi}{L})^2t),
\end{align*}
donde $b=c_2c_3$.

Si tomamos una de estas soluciones y le aplicamos la condición
inicial $u(x,0)=f(x)$, obtenemos la ecuación
\[
  f(x) = b \sin(\tfrac{n\pi}{L}x)
,\]
la cual no se puede satisfacer para cualquier función $f(x)$.

Sin embargo, si tomamos una constante $b_n$ para cada $n$ y
sumamos las soluciones $u_n(x,t)$ resultantes, obtenemos
\begin{equation}
  u(x,t)
  = \sum_{n=1}^{\infty}
  b_n\sin(\tfrac{n\pi}{L}x)\exp(-k(\tfrac{n\pi}{L})^2t),
\end{equation}
la cual sigue siendo una solución de la ecuación diferencial, ya
que la linealidad de la E.D. nos permite aplicar
el principio de superposición.

Ahora sí, aplicando la condición inicial $u(x,0)=f(x)$, tenemos
\[
  f(x)
  = \sum_{n=1}^{\infty}
  b_n\sin(\tfrac{n\pi}{L}x),
,\]
lo cual tiene solución exactamente cuando $f$ se puede expresar
como una serie de Fourier en senos en $[0,L]$. En este caso, los
coeficientes son
\[
  b_n = \frac{2}{L}\int_0^L f(x)\sin(\tfrac{n\pi}{L}x)\,dx
.\]
En particular, una condición necesaria para que $f$ se pueda
expresar de esta manera es la llamada condición de
compatibilidad:
\[
  f(0)=f(L)=0
.\]

\begin{example}
  Considérese una varilla de longitud $L=\pi$ y
  \begin{align*}
    f(x)&=x(\pi-x)
    &
    k&=1.
  \end{align*}
  Notemos que $f$ cumple las condiciones de compatibilidad.

  Además, $f$ tiene transformada de Fourier en senos en
  $[0,\pi]$, así que la solución está dada por
  \[
    u(x,t)
    = \sum_{n=1}^{\infty} b_n\sin(nx)\exp(-kn^2t),
  .\]
  donde
  \begin{align*}
    b_n
    &= \frac{2}{\pi}\int_0^\pi x(\pi-x)\sin(nx)\,dx
  \end{align*}
  Empleando el método tabular
  \[
    \begin{array}{c|c|c|c}
      & u & &  dv \\
      (+) & x(\pi-x) & & \sin(nx) \\
      && \searrow \\
      (-) & \pi - 2x && -\frac{\cos(nx)}{n} \\
      && \searrow \\
      (+) & -2 && -\frac{\sin(nx)}{n^2}\\
      && \searrow \\
      (-) & 0 && \frac{\cos(nx)}{n^3}
    \end{array}
  .\]
  Obtenemos
  \begin{align*}
    b_n
    &= \frac{2}{\pi}\left[ -x(\pi-x)\frac{\cos(nx)}{n}
    + (\pi-2x)\frac{\sin(nx)}{n^2}
    - \frac{2\cos(nx)}{n^3}
    \right]_{x=0}^{x=\pi} \\
    &= \frac{2}{\pi}\left[
    -\frac{2\cos(n\pi)}{n^3}+\frac{2\cos(n0)}{n^3}
    \right] \\
    &= \frac{4(1-(-1)^n)}{n^3\pi} \\
    &=
    \begin{cases}
      \displaystyle\frac{8}{n^3\pi} & n \text{ impar } \\
      0 & n \text{ par.}
    \end{cases}
  \end{align*}
  o bien
  \begin{align*}
    b_{2n-1} &= \frac{8}{(2n-1)^3\pi}
    &
    b_{2n}&=0
  \end{align*}
  para $n=1,2,3,\dots$. Así,
  \begin{align*}
    u(x,t)
    &= \frac{8}{\pi}\sum_{n=1}^{\infty}
    \frac{1}{(2n-1)^3}\sin((2n-1)x)\exp(-k(2n-1)^2t).
  \end{align*}
\end{example}

\subsubsection{Extremos aislados}
Consideremos ahora el problema de la distribución de temperatura
sobre una varilla de longitud $L$ donde los extremos son aislados
con temperatura inicial dada por $f(x)$ a lo largo de la varilla

\paragraph{Modelo}
Sea $u=u(x,t)$ la función deseada. La ecuación diferencial
\[
  \frac{\partial u}{\partial u}
  =k
  \frac{\partial^2u}{\partial^2x}
  \hspace{10mm} 0<x<L, t>0
.\]
\begin{align*}
  C.F. && u_x(0,t) &= 0 & u_x(L,t) &= 0 &t&>0 \\
  C.I. && u(x,0) &= f(x) & &&& 0<x<L
\end{align*}
Queda de tarea.

\subsubsection{Extremos con temperatura constante (mie 17 nov
2021)}

Determinar la función que determina la distribución de
temperatura sobre una varilla de longitud $L$ laterlamente
aislada en donde los extremos tienen temperatura constante en el
extremo izquierdo $T_1$ y en el extremo derecho $T_2$, con
temperatura inicial dada por $f(x), 0\leq x\leq L$.

\paragraph{Modelo}
Sea $u=u(x,t)$. Entonces la EDP es
\[
  \frac{\partial u}{\partial t}
  = k
  \frac{\partial ^2 u}{\partial x^2}
  \hspace{10mm} 0<x<L,
  \hspace{5mm} t>0.
\]
con
\begin{align*}
  C.F. && u(0,t) &= T_1 & u(L,t) &= T_2 &t&>0 \\
  C.I. && u(x,0) &= f(x) & &&& 0\leq x\leq L
\end{align*}

\paragraph{Solución}
Obsevemos que, si $v$ es una solución del problema con
temperatura cero en los extremos
\[
  \frac{\partial v}{\partial t}
  = k
  \frac{\partial ^2 v}{\partial x^2}
  \hspace{10mm} 0<x<L,
  \hspace{5mm} t>0.
\]
con
\begin{align*}
  C.F. && v(0,t) &= 0 & v(L,t) &= 0 &t&>0 \\
  C.I. && v(x,0) &= g(x) & &&& 0\leq x\leq L
\end{align*}
entonces $u(x,t)=v(x,t)+a_1+a_2x$ es solución de
\[
  \frac{\partial u}{\partial t}
  = k
  \frac{\partial ^2 u}{\partial x^2}
  \hspace{10mm} 0<x<L,
  \hspace{5mm} t>0.
\]
con
\begin{align*}
  C.F. && u(0,t) &= a_1 & u(L,t) &= a_1+a_2L &t&>0 \\
  C.I. && u(x,0) &= g(x)+a_1+a_2x & &&& 0\leq x\leq L
\end{align*}
Por lo tanto, podemos tomar $a_1=T_1$, $a_2=(T_2-T_1)/L$ y
$g(x)=f(x)-a_1-a_2x$ para reducir este problema al problema
anterior, obteniendo

\[
  u(x,t) = v(x,t) + T_1 + \frac{T_2-T_1}{L}x
\]
donde
\begin{equation}
  v(x,t)
  = \sum_{n=1}^{\infty}
  b_n\sin(\tfrac{n\pi}{L}x)\exp(-k(\tfrac{n\pi}{L})^2t),
\end{equation}

\[
  g(x)
  = f(x)-T_1-\frac{T_2-T_1}{L}
  = \sum_{n=1}^{\infty}
  b_n\sin(\tfrac{n\pi}{L}x),
,\]
lo cual tiene solución exactamente cuando los coeficientes de
Fourier
\[
  b_n = \frac{2}{L}\int_0^L
    \left[f(x) - T_1 - \frac{T_2-T_1}{L}x\right]
    \sin(\tfrac{n\pi}{L}x)\,dx
\]
existen. En particular, es necesario que se cumpla
la condición de compatibildad
\begin{align*}
  f(0) &= T_1 & f(L) &= T_2.
\end{align*}

\begin{example}
  Caso particular: $L=\pi$, $f(x)=x$, $T_1=50$, $T_2=100$.

  Entonces
  \begin{align*}
    b_n
    = \frac{2}{\pi}\int_0^\pi(x-50-\tfrac{50}{\pi}x)\sin(nx)\,dx.
  \end{align*}
  Usando el método tabular
  \[
    \begin{array}{c|c|c|c}
      & u & &  dv \\
      (+) & x-50-\tfrac{50}{\pi}x & & \sin(nx) \\
      && \searrow \\
      (-) & 1-\tfrac{50}{\pi} && -\frac{\cos(nx)}{n} \\
      && \searrow \\
      (+) & 0 && -\frac{\sin(nx)}{n^2}\\
    \end{array}
  .\]
  tenemos
  \begin{align*}
    b_n
    &= \frac{2}{\pi}\int_0^\pi(x-50-\tfrac{50}{\pi}x)\sin(nx)\,dx
    \\
    &= \frac{2}{\pi}\left[
    -\left(x-50-\frac{50}{\pi}x\right)\frac{\cos(nx)}{n}
    + \left(1-\frac{50}{\pi}\right)\frac{\sin(nx)}{n^2}
    \right]_{x=0}^\pi \\
    &= \frac{2}{\pi}
    \left[
    (100-\pi)\frac{\cos(n\pi)}{n}
    \right]
    -
    \frac{2}{\pi} \frac{50}{n} \\
    &= \frac{2}{\pi}\frac{(100-\pi)(-1)^n-50}{n}.
  \end{align*}
  Luego,
  \begin{align*}
    u(x,t)
    &= v(x,t) + 50+\frac{50}{\pi}x \\
    &= 50+\frac{50}{\pi}x
      + \frac{2}{\pi}\sum_{n=1}^{\infty}
      \frac{(100-\pi)(-1)^n-50}{n}
      \sin(nx)\exp(-kn^2t),
  \end{align*}
\end{example}

\subsubsection{Modelo con extremos que irradian temperatura}

Si en el problema anterior las condiciones de frontera fueran de
la forma
\begin{align*}
  u_x(0,t) &= A[u(0,t) + T] \\
  u_x(L,t) &= -A[u(L,t) + T]
\end{align*}
se dice que hay irradiación de temperatura en los extremos.

\paragraph{Modelo}
Resolveremos el problema en el cual la temperatura en el extremo
izquier\-do es cero y en el derecho hay irradiación tomando $T=0$.
Esto es
\[
  \frac{\partial u}{\partial t}
  = k
  \frac{\partial ^2 u}{\partial x^2}
  \hspace{10mm} 0<x<L,
  \hspace{5mm} t>0.
\]
con
\begin{align*}
  C.F. && u(0,t) &= 0 & u_x(L,t) + Au(L,t) &= 0 &t&>0 \\
  C.I. && u(x,0) &= f(x) & &&& 0\leq x\leq L
\end{align*}
\paragraph{Solución}
Separando variables con el supuesto $u(x,t)=X(x)T(t)$,
obtenemos las ecuaciones
\begin{align*}
  X''+\lambda X &= 0
  &
  T'+k\lambda T &= 0.
\end{align*}

Aplicando las condiciones de frontera, tenemos
\begin{align*}
  \text{Si } u(0,t)&=0
  & &\implies &
  X(0)T(t)&=0 \\
  && &\text{ ssi } &
  X(0)&=0 \\
  \text{Si } u_x(L,t)+A(L,t)&=0
  & &\implies &
  X'(L)T(t)+AX(L)T(t)&=0 \\
  && &\text{ ssi } &
  X'(L)+AX(L)&=0.
\end{align*}
Así, obtenemos un problema de Strum-Liouville completo:
\begin{align*}
  X''+\lambda X &= 0 \\
  X(0) &= 0 \\
  X'(L) + AX(L) &= 0.
\end{align*}
\begin{itemize}
  \item Caso $\lambda=0$.
  Entonces $X$ tiene solución
  \[
    X(x) = c_1 + c_2x
  .\]
  Aplicando las condiciones de frontera, tenemos
  \begin{align*}
    X(0)&=0
    &\implies &&
    c_1&=0
    \\
    X'(L)+AX(L)&=0
    &\implies &&
    c_2+Ac_2L&=0
    &\implies &&
    c_2 &= 0,
  \end{align*}
  por lo cual la solución es trivial.

  \item Caso $\lambda<0$.
  Entonces
  \[
    X(x) = c_1e^{\alpha x} + c_2e^{-\alpha x}
  \]
  con $\alpha=\sqrt{-\lambda}$.

  \item Caso $\lambda>0$.
  Entonces
  \[
    X(x) = c_1\cos(\beta x) + c_2\sin(\beta x)
  \]
  donde $\beta = \sqrt\lambda$.
  Aplicando $X(0)=0$, obtenemos $c_1=0$, por lo cual
  \begin{align*}
    X(x) &= c_2\sin(\beta x) \\
    X'(x) &= \beta c_2\cos(\beta x)
  \end{align*}
  Aplicando la otra condición de frontera, tenemos
  \begin{align*}
    \beta c_2\cos(\beta L)+Ac_2\sin(\beta L) &= 0.
  \end{align*}
  Suponiendo $c_2\neq 0$ (de otro modo la solución es trivial),
  tenemos
  \begin{align*}
    \beta\cos(\beta L)+A\sin(\beta L) &= 0.
  \end{align*}
  Esta ecuación no se puede resolver algebraicamente para
  $\beta$, pero sí tiene infinitas soluciones
  $\beta_1,\beta_2,\beta_3,\beta_4,\dots$.
  Luego, tenemos valores para $\lambda$ dados como
  $\lambda_n=\beta_n^2$, lo cual nos da soluciones
  \[
    X(x) = b\sin(\beta_n x)
  \]
  con $n\in\N$ y $c\in\R$.

  Así,
  \[
    T(t) = b\exp(-k\beta_n^2 t)
  \]
  por lo cual obtenemos soluciones
  \[
    u(x,t)
    =b\sin(\beta_n x)\exp(-k\beta_n^2 t)
  .\]
\end{itemize}

\begin{example}
  Considérese el caso con $L=6$, $k=4$, $A=\frac{1}{2}$ y $f(x)=x(6-x)$.
\end{example}

\subsubsection{Problema de calor no homogéneo}

Consideremos el problema de distribución de temperatura sobre una varilla
delgada de longitud $L$ lateralmente aislada con temperatura cero en los
extremos y temperatura inicial dada por $f(x)$ para $0\leq x\leq L$ y en el cual
existe una fuerza externa dada por $F(x,t)$.

\paragraph{Modelo}

Sea $u=u(x,t)$ la temperatura. La ecuación diferencial que gobierna a este
fenómeno es
\begin{equation}\label{eqn:uno}
  \frac{\partial u}{\partial x}
  =k \frac{\partial^2 u}{\partial x^2} + F(x,t)
\end{equation}
\begin{align*}
  C.F. && u(0,t) &= 0, & u(L,t)&= 0 && t>0 \\
  C.I. && u(x,0) &= f(x), &&&& 0\leq x\leq L
\end{align*}

\paragraph{Solución}
Dado que la ecuación es lineal y no homogénea, la solución general
está dada como
\[
  u(x,t) = u_h(x,t)+u_p(x,t)
\]
donde $u_p(x,t)$ es una solución particular y
$u_h(x,t)$ es la solución de la ecuación homogénea correspondiente
\begin{align*}
  && \frac{\partial u}{\partial x^2}
  &=k \frac{\partial^2 u}{\partial x^2} \\
  C.F. && u(0,t) &= 0, & u(L,t)&= 0 && t>0 \\
  C.I. && u(x,0) &= f(x), &&&& 0\leq x\leq L.
\end{align*}
Recordemos que este modelo homogéneo tiene solución
\[
  u(x,t)
  =
  \sum_{n=1}^{\infty}b_n\sin(\tfrac{n\pi}{L}x)
  \exp(-k(\tfrac{n\pi}{L})^2t)
\]
con
\[
  b_n=\frac{2}{L}\int_{0}^{L}f(x)\sin(\tfrac{n\pi}{L}x)\,dx
.\]
Esto sugiere que, para determinar la solución de \eqref{eqn:uno},
es razonable suponer que $u$ tiene la forma
\begin{equation}\label{eqn:dos}
  u(x,t) = \sum_{n=1}^{\infty}T_n(t)\sin(\tfrac{n\pi}{L}x),
\end{equation}
donde $T_n(t)$ es una función desconocida que, en el caso homogéneo,
se reduce a $T_n(t)=b_n\exp(-k(\tfrac{n\pi}{L})^2t)$).
La ecuación \eqref{eqn:dos} representa la serie en senos de $u(x,t)$,
por lo cual tenemos
\begin{equation}\label{eqn:tres}
  T_n(t) = \frac{2}{L} \int_{0}^{L}u(x,t)\sin(\tfrac{n\pi}{L}x)\,dx.
\end{equation}
Ahora supongamos que $F(x,t)$ tiene serie de Fourier en senos
\begin{align*}
  F(x,t)
  &= \sum_{n=1}^{\infty}B_n(t)\sin(\tfrac{n\pi}{L}x),
  &B_n(t)
  &= \frac{2}{L}\int_{0}^{L}F(x,t)\sin(\tfrac{n\pi}{L}x)\,dx.
\end{align*}
En este caso, al derivar \eqref{eqn:tres} con respecto a $t$, obtenemos
\begin{align*}
  T'_n(t)
  &=
  \frac{2}{L}\int_{0}^{L}\frac{\partial u}{\partial t}(x,t)
  \sin(\tfrac{n\pi}{L}x)\,dx \\
  &=
  \frac{2}{L}\int_{0}^{L}
  \left[
  k \frac{\partial^2u}{\partial x^2}(x,t)+F(x,t)
          \right]
  \sin(\tfrac{n\pi}{L}x)\,dx \\
  &=
  \frac{2k}{L}\int_{0}^{L}
  \frac{\partial^2u}{\partial x^2}(x,t)
  \sin(\tfrac{n\pi}{L}x)\,dx
  +
  \frac{2}{L}\int_{0}^{L}
  F(x,t)
  \sin(\tfrac{n\pi}{L}x)\,dx \\
  &=
  \frac{2k}{L}\left[
    \frac{\partial u}{\partial x}(x,t)\sin(\tfrac{n\pi}{L}x)\Big|_{x=0}^L
    -
    \frac{n\pi}{L}\int_{0}^{L}\frac{\partial u}{\partial x}(x,t)
    \cos(\tfrac{n\pi}{L}x)\,dx
  \right]
  +B_n(t) \\
  &=
  -\frac{2kn^2\pi^2}{L^3}\int_{0}^{L}u(x,t)\sin(\tfrac{n\pi}{L}x)\,dx
  +B_n(t) \\
  &= -k(\tfrac{n\pi}{L})^2T_n(t)+B_n(t).
\end{align*}
Es decir,
\[
  T'_n(t)
  +k(\tfrac{n\pi}{L})^2T_n(t)
  =B_n(t)
,\]
lo cual es una EDO lineal de primer orden, cuya solución es
\[
  T_n(t)
  =
  \exp(-k(\tfrac{n\pi}{L})^2t)
  \left[
  \int_{0}^{t}\exp(k(\tfrac{n\pi}{L})^2\xi)
  B_n(\xi)\,d\xi + C\right]
.\]
En $t=0$ tenemos que $T_n(0)=C$ es una constante, pero $T_n(0)=b_n$,
así que $C=b_n$.

Así, sustituyendo en \eqref{eqn:dos}, tenemos
\[
  u(x,t)
  =
  \sum_{n=1}^{\infty}\exp(-k(\tfrac{n\pi}{L})^2t)
  \left[
    \int_{0}^{t}\exp(k(\tfrac{n\pi}{L})^2\xi)B_n(\xi)\,d\xi + b_n
  \right]\sin(\tfrac{n\pi}{L}x)
.\]
Es decir,
\begin{align*}
  u(x,t)
  &=
  \sum_{n=1}^{\infty}\exp(-k(\tfrac{n\pi}{L})^2t)
  \int_{0}^{t}\exp(k(\tfrac{n\pi}{L})^2\xi)
  B_n(\xi)\,d\xi\sin(\tfrac{n\pi}{L}x) \\
  &\hspace{10mm} +
  \sum_{n=}^{\infty}b_n\sin(\tfrac{n\pi}{L}x)
  \exp(-k(\tfrac{n\pi}{L})^2t)
\end{align*}

\begin{exe}
  Consideremos el caso particular de $L=5$, $k=1$, $f(x)=x$, $F(x,t)=x\sin t$.
\end{exe}

\subsection{Problemas de vibraciones}

Considérese el problema de determinar las vibraciones en una varilla
de longitud $L$ fija en los extremos con un desplazamiento inicial dado
por $f(x)$ y una velocidad inicial dada por $g(x)$ para $0\leq x\leq
L$.

Primero trataremos el modelo ideal en que la cuerda o varilla no tiene
nada que lo impida.

\paragraph{Modelo}
Sea $u=u(x,t)$ la función que describa las vibraciones. Entonces la
ecuación diferencial que domina la evolución del sistema es la
ecuación de onda:

\[
    \frac{\partial^2u}{\partial t^2 }
    =c^2   \frac{\partial^2u}{\partial x^2}
    \quad 0<x<L \quad t>0
\]
con
\begin{align*}
  C.F.&& u(0,t)&=0 & u(L,t)&=0 & t>0 \\
  C.I.&& u(x,0)&=f(x) & u_t(x,0)&=g(x) & 0\leq x\leq L.
\end{align*}

\paragraph{Solución}
El método de separación de variables supone que la solución $u$ es de
la forma $u(x,t)=X(x)T(t)$. Sustituyendo en la ecuación, obtenemos
\[
  XT''=c^2X''T
,\]
de donde
\[
  \frac{T''}{c^2T} = \frac{X''}{X} = -\lambda
,\]
por lo cual obtenemos las EDO
\begin{align*}
  X''+\lambda X &= 0
  &
  T''+c^2\lambda T &= 0.
\end{align*}
Ahora apliquemos las condiciones
\begin{itemize}
  \item Dado que $u(0,t)=0$, entonces $X(0)T(t)=0$, por lo cual
    $X(0)=0$.
  \item Dado que $u(L,t)=0$, entonces $X(L)T(t)=0$, por lo cual
    $X(L)=0$.
\end{itemize}
Así, obtenemos el problema de Strum-Liouville
\[
  X''+\lambda X = 0, \quad X(0)=X(L)=0
\]
que tiene un espacio de soluciones generado por
\[
  X(x)=c\sin(\tfrac{n\pi}{L}x), \quad \text{ con }
  \lambda=(\tfrac{n\pi}{L})^2
\]
y $c\in\R$.
Luego, la solución de $T''+(\tfrac{cn\pi}{L})^2T=0$ es 
\[
  T(t)=k_1\cos(\tfrac{cn\pi}{L}t)+k_2\sin(\tfrac{cn\pi}{L}t)
\]
donde $k_1,k_2\in\R$.
Luego, tenemos soluciones
\[
  u_n(x,t)=\sin(\tfrac{n\pi}{L}x) \left[
    a\cos(\tfrac{cn\pi}{L}t)+b\sin(\tfrac{cn\pi}{L}t)
  \right]
.\]
Tomando coeficientes distintos $a_n,b_n$ para cada $n$ y sumando,
obtenemos una solución
\[
  u(x,t)
  =
  \sum_{n=1}^{\infty}\sin(\tfrac{n\pi}{L}x) \left[
    a_n\cos(\tfrac{cn\pi}{L}t)
    +b_n\sin(\tfrac{cn\pi}{L}t)
  \right]
\]
con derivada
\[
  \frac{\partial u}{\partial t}
  =
  \sum_{n=1}^{\infty}\sin(\tfrac{n\pi}{L}x) \left[
    -a_n \tfrac{cn\pi}{L}\sin(\tfrac{cn\pi}{L}t)
    +b_n \tfrac{cn\pi}{L}\cos(\tfrac{cn\pi}{L}t)
  \right]
.\]
Al aplicar las condiciones iniciales, tenemos
\begin{itemize}
  \item Como $u(x,0)=f(x)$, entonces
    \[
      f(x)=\sum_{n=1}^{\infty}a_n\sin(\tfrac{n\pi}{L}x)
    .\]
  \item Como $u_t(x,0)=g(x)$, entonces
    \[
      g(x) = \sum_{n=1}^{\infty}b_n
      \tfrac{cn\pi}{L}\sin(\tfrac{n\pi}{L}x)
    .\]
\end{itemize}
Estas dos ecuaciones tienen solución cuando $f$ y $g$ tienen serie de
Fourier en senos, en cuyo caso los
coeficientes se pueden obtener como
\begin{align*}
  a_n &= \frac{2}{l}\int_{0}^{L}f(x)\sin(\tfrac{n\pi}{L}x)\,dx
      &
  b_n \tfrac{cn\pi}{L} &=
  \frac{2}{L}\int_{0}^{L}g(x)\sin(\tfrac{n\pi}{L}x)\,dx.
\end{align*}
Es decir,
\begin{align*}
  a_n &= \frac{2}{l}\int_{0}^{L}f(x)\sin(\tfrac{n\pi}{L}x)\,dx
      &
  b_n &=
  \frac{2}{cn\pi}\int_{0}^{L}g(x)\sin(\tfrac{n\pi}{L}x)\,dx.
\end{align*}

\begin{example}
  Consideremos el caso particular con $L=\pi$, $c=1$ y condiciones
  iniciales
  \begin{align*}
    f(x) &=
    \begin{cases}
      x, &0\leq x\leq \frac{\pi}{2} \\
      \pi-x, & \frac{\pi}{2}<x\leq\pi.
    \end{cases} \\
    g(x) &= x(1+\cos x).
  \end{align*}.

\paragraph{Solución}

  La solución es
  \[
    u(x,t)
    =
    \sum_{n=1}^{\infty}\sin(nx)
    \left[ a_n\cos(cnt) +b_n\sin(cnt) \right]
  \]
  con
  \begin{align*}
    a_n &= \frac{2}{\pi}\int_{0}^{\pi}f(x)\sin(nx)\,dx, \\
    b_n &= \frac{2}{cn\pi}\int_{0}^{\pi}g(x)\sin(nx)\,dx.
  \end{align*}

  Tenemos
  \begin{align*}
    a_n
    &= \frac{2}{\pi}\int_{0}^{\pi}f(x)\sin(nx)\,dx \\
    &= \frac{2}{\pi}\int_{0}^{\pi /2} x\sin(nx)\,dx
     + \frac{2}{\pi}\int_{\pi /2}^{\pi}(\pi-x)\sin(nx)\,dx \\
    &= \frac{2}{\pi n}\int_{0}^{\pi /2} \cos(nx)\,dx
      - \frac{2x\cos(nx)}{\pi n} \Big|_{0}^{\pi /2}
     \\ & \hspace{10mm}
     - \frac{2}{\pi n}\int_{\pi /2}^{\pi}\cos(nx)\,dx
     - \frac{2(\pi-x)\cos(nx)}{\pi n} \Big|_{\pi /2}^{\pi} \\
    &= \frac{2}{\pi n^2} \sin(nx) \Big|_{0}^{\pi /2}
      - \frac{\cos(n\pi /2)}{n}
     \\ & \hspace{10mm}
     - \frac{2}{\pi n^2}\sin(nx)\Big|_{\pi /2}^{\pi}
     + \frac{\cos(n\pi /2)}{n} \\
    &= \frac{2}{\pi n^2} \sin(nx) \Big|_{0}^{\pi /2}
     - \frac{2}{\pi n^2}\sin(nx)\Big|_{\pi /2}^{\pi} \\
    &= \frac{4}{\pi n^2}\sin(n\pi /2).
  \end{align*}

  Y haciendo la otra integral (yo la hice con Wolfram) obtenemos
  \begin{align*}
    b_1 &= \frac{3}{2} \\
    b_n
    &= \frac{2(-1)^n}{n^2(n^2-1)}, \quad n\geq 2.
  \end{align*}
\end{example}
Así,
\begin{align*}
  u(x,t)
  &= \sin(x)[a_1\cos(t)+b_1\sin(t)]
  \\ &\hspace{10mm}
  + \sum_{n=2}^{\infty}\sin(nx)
    \left[
      a_n\cos(nt)+b_n\sin(nt)
    \right] \\
  &= \sin(x)[\tfrac{4}{\pi}\cos(t)+\tfrac{3}{2}\sin(t)]
  \\ &\hspace{10mm}
  + \sum_{n=2}^{\infty}\sin(nx)
    \left[
      \frac{4}{n^2\pi}\cos(nt)+\frac{2(-1)^n}{n^2(n^2-1)}\sin(nt)
    \right] \\
\end{align*}

\subsubsection{Vibraciones con perturbación}
Consideremos el problema anterior en donde ahora existe una fuerza
externa $h(x)$ que impide que la cuerda vibre libremente.

\paragraph{Modelo}
La ecuación diferencial que gobierna a este fenóneno está dada como
\[
  \frac{\partial ^2u}{\partial t^2}
  = c^2 \frac{\partial ^2u}{\partial x^2} + h(x)
\]
con
\begin{align*}
  C.F.&& u(0,t)&=0 & u(L,t)&=0 & t>0 \\
  C.I.&& u(x,0)&=f(x) & u_t(x,0)&=g(x) & 0\leq x\leq L.
\end{align*}

\paragraph{Solución}
El método de separación de variables supone que la solución es de la
forma $u(x,t)=X(x)T(t)$. Sustituyendo esto en la ecuación, tenemos
\[
  XT''=c^2X''T+h(x)
\]
dividiendo entre $XT$, tenemos
\[
  \frac{T''}{T} = c^2 \frac{X''}{X} + \frac{h(x)}{XT}
.\]
Así, las variables no se pueden separar.

Para resolver esto, introducimos una perturbación $\phi$ para
considerar soluciones de la forma $u(x,t)=v(x,t)+\phi(x)$.
Sustituyendo en la EDP, tenemos
\[
  \frac{\partial ^2v}{\partial t^2}
  = c^2 \frac{\partial ^2v}{\partial x^2} + c^2\phi''(x)+ h(x)
.\]
Supongamos que podemos resolver las ecuaciones
\begin{align*}
  c^2\phi''+h &= 0 \\
  \frac{\partial ^2v}{\partial t^2}
  &= c^2 \frac{\partial ^2v}{\partial x^2}.
\end{align*}
Donde las condiciones para $v$ y de $\phi$ se obtienen de las de $u$:
\begin{itemize}
  \item Como $u(0,t) = 0$, entonces $v(0,t)+\phi(0)=0$, por lo cual
    $v(0,t)=0$ y $\phi(0)=0$.
  \item Como $u(L,t)=0$, entonces $v(L,t)+\phi(L)=0$, por lo cual
    $v(L,t)=0$ y $\phi(L)=0$.
  \item Como $u(x,0)=f(x)$, entonces
    $v(x,0)=u(x,0)-\phi(x)=f(x)-\phi(x)$.
  \item Como $u_t(x,0)=g(x)$, entonces $v_t(x,0)=g(x)$.
\end{itemize}
Por lo cual $v$ es la solución del problema
\[
  \frac{\partial ^2v}{\partial t^2}
  = c^2 \frac{\partial ^2v}{\partial x^2}
\]
con
\begin{align*}
  C.F.&& v(0,t)&=0 & v(L,t)&=0 & t>0 \\
  C.I.&& v(x,0)&=f(x)-\phi(x) & v_t(x,0)&=g(x) & 0\leq x\leq L.
\end{align*}

Además, la ecuación $c^2\phi''+h=0$ tiene solución general
\[
  \phi(x) = -\frac{1}{c^2}\int_{0}^{x}H(\xi)\,d\xi+c_1x+c_2,
  \quad H(\xi) = \int_{0}^{x}h(\xi)\,d\xi
.\]
Por lo tanto,
\[
  \phi(x)=-\frac{1}{c}\int_{0}^{x}H(\xi)\,d\xi
  + \frac{x}{c^2L}\int_{0}^{L}H(\xi)\,d\xi
.\]

\begin{exe}
  Consideremos el caso particular $L=\pi$, $c=1$, $h(x)=ax$, $f(x)=0$
  y $g(x)=x$.
\end{exe}

\subsection{Vibraciones en dos dimensiones}

\subsubsection{Vibraciones en una membrana rectangular}

\subsubsection{Vibraciones en una membrana circular}
Supongamos que se desea determinar la función que describe las
vibraciones sobre una membrana circular de radio $r=a$ fija a lo largo
de toda la frontera.
Dado que la ecuación de onda está dada en la forma
\[
  \frac{\partial^{2} u}{\partial t^{2}}
  = c^{2} \left(
    \frac{\partial ^{2}u}{\partial x^{2}}
    +\frac{\partial ^{2}u}{\partial y^{2}}
  \right)
.\]
debemos expresarla en coordenadas polares:
\begin{align*}
  x &= r\cos\theta \\
  y &= r\sin\theta \\
\end{align*}
por lo cual
\begin{align*}
  r^2 &= x^{2}+y^{2} \\
  \theta &= \tan^{-1}(\tfrac{y}{x})
\end{align*}
Por la regla de la cadena, tenemos
\begin{align*}
  \begin{bmatrix}
    \frac{\partial u}{\partial x} \\[1mm]
    \frac{\partial u}{\partial y}
  \end{bmatrix}
  &=
  \begin{bmatrix}
    \frac{\partial r}{\partial x} &
    \frac{\partial \theta}{\partial x} \\[1mm]
    \frac{\partial r}{\partial y} &
    \frac{\partial \theta}{\partial y}
  \end{bmatrix}
  \begin{bmatrix}
    \frac{\partial u}{\partial r} \\[1mm]
    \frac{\partial u}{\partial \theta}
  \end{bmatrix}
  \\
  Hu
  &=
  \begin{bmatrix}
    \frac{\partial r}{\partial x} & \frac{\partial r}{\partial y}
  \end{bmatrix}
  \begin{bmatrix}
    \frac{\partial ^{2}u}{\partial r^{2}} & 
    \frac{\partial ^{2}u}{\partial r\partial\theta} \\[1mm]
    \frac{\partial ^{2}u}{\partial \theta\partial r} & 
    \frac{\partial ^{2}u}{\partial \theta^{2}}
  \end{bmatrix}
\end{align*}

\begin{align*}
  \frac{\partial^{2} u}{\partial x^{2}}
  +
  \frac{\partial^{2} u}{\partial y^{2}}
  &=
  \frac{\partial u}{\partial r^{2}}
  +
  \frac{1}{r^{2}}\frac{\partial u}{\partial \theta^{2}}
  +
  \frac{1}{r}\frac{\partial u}{\partial r}
\end{align*}
Así, la ecuación queda como
\[
  \frac{\partial u}{\partial t^{2}}
  =c^{2}
  \left(
    \frac{\partial u}{\partial r^{2}}
    +
    \frac{1}{r^{2}}\frac{\partial u}{\partial \theta^{2}}
    +
    \frac{1}{r}\frac{\partial u}{\partial r}
  \right)
.\]
con $u=u(r,\theta,t)$.
Como la membrana está fija en la frontera, tenemos
\begin{align*}
  u(0,\theta,t)&= 0 & u(a,\theta,t)&= 0 \\
  u(r,\pi,t)&=u(r,-\pi,t) & u_\theta(r,\pi,t)&=u_\theta(r,-\pi,t).
\end{align*}
Consideremos también que el desplazamiento inicial está dado por
$f(r,\theta)$ y la velocidad inicial por $g(r,\theta)$ para todo
$0\leq r\leq a$ y para $-\pi\leq\theta\leq\pi$.

\paragraph{Solución}
Por separación de variables.
Sea $u(r,\theta,t)=R(r)\Phi(\theta)T(t)$.
Sustituyendo en la ecuación, tenemos
\[
  R\Phi T''
  = c^{2}
  (R''\Phi T
  +\frac{1}{r}R'\Phi T
  +\frac{1}{r^{2}}R\Phi''T)
.\]
Dividiendo entre $u$, esto es
\[
  \frac{T''}{c^{2}T}
  =
  \left(
    \frac{R''}{R}
    +\frac{1}{r}\frac{R'}{R}
    +\frac{1}{r^{2}}\frac{\Phi''}{\Phi}
  \right)
  =-\lambda^{2}
.\]
De aquí, obtenemos las EDO
\begin{align*}
  T''+\lambda^{2}c^{2}T &= 0 \\
  r^{2}\frac{R''}{R}
  +r\frac{R'}{R}
  +r^{2}\lambda^{2}
  &=
  -\frac{\Phi''}{\Phi}
  = \mu
\end{align*}
Esta última ecuación se puede escindir en otras dos ecuaciones, para
obtener el sistema de tres EDO:
\begin{align*}
  T''+\lambda^{2}c^{2}T &= 0 \\
  r^{2}R'' +rR' +(r^{2}\lambda^{2}-\mu)R &= 0 \\
  \Phi'' +\mu\Phi &= 0.
\end{align*}
Aplicando las condiciones de frontera
\begin{itemize}
  \item
    $u(0,\theta,t)=0$ implica que $R(0)=0$.
  \item
    $u(a,\theta,t)=0$ implica que $R(a)=0$.
  \item
    $u(r,\pi,t)=u(r,-\pi,t)$ implica que $\Phi(\pi)=\Phi(-\phi)$.
  \item
    $u_\theta(r,\pi,t)=u_\theta(r,-\pi,t)$ implica que
    $\Phi'(\phi)=\Phi'(-\pi)$.
\end{itemize}
Resolviendo para $\Phi$, tenemos
\begin{itemize}
  \item
    Si $\mu=0$, entonces $\Phi(\theta)=c_1+c_2\theta$, así que
    $\Phi'(\theta)=c_2$.

    Como $\Phi(\pi)=\Phi(-\pi)$, entonces $c_1+c_2\pi=c_1-c_2\pi$,
    por lo cual $c_2=0$.
    
    Como $\Phi'(\pi)=\Phi'(-\pi)$, entonces $c_2=c_2$. Por
    lo tanto, $\Phi(\theta)=c_1$.
  
  \item
    Si $\mu<0$, entonces
    \[
       \Phi(\theta)
       =c_1e^{\alpha\theta}+
       c_2e^{-\alpha\theta}
    \]
    donde $\alpha=\sqrt{-\mu}$.

    Como $\Phi(\pi)=\Phi(-\pi)$, entonces
    $c_1e^{\alpha\pi}+c_2e^{-\alpha\pi}
    =c_1e^{-\alpha\pi}+c_2e^{\alpha\pi}$. Se sigue que
    $c_1=c_2$.

    Como $\Phi'(\pi)=\Phi'(-\pi)$, entonces
    $\alpha c_1e^{\alpha\pi}+\alpha c_2e^{-\alpha\pi}
    = \alpha c_1e^{-\alpha\pi}-\alpha c_2e^{\alpha\pi}$. Se
    sigue que $c_1=-c_2$. Luego, $c_1=c_2=0$.
    
    Concluimos que la única solución es la trivial.

  \item
    Si $\mu>0$. Entonces
    \begin{align*}
      \Phi(\theta) &=
      c_1\cos(\beta\theta)+c_2\sin(\beta\theta) \\
      \Phi'(\theta) &=
      -\beta c_1\sin(\beta\theta)+\beta c_2\cos(\beta\theta)
    \end{align*}
    donde $\beta=\sqrt\mu$.
    
    Como $\Phi(\pi)=\Phi(-\pi)$, entonces
    \[
      c_1\cos(\beta\pi)+c_2\sin(\beta\pi)
      =
      c_1\cos(-\beta\pi)+c_2\sin(-\beta\pi)
    .\]
    Luego, $\sin(\beta\pi)=0$.

    Como $\Phi(\pi)=\Phi(-\pi)$, entonces
    \[
      -\beta c_1\sin(\beta\pi)+\beta c_2\cos(\beta\pi)
      =
      -\beta c_1\sin(-\beta\pi)+\beta c_2\cos(-\beta\pi)
    .\]
    Luego, $\sin(\beta\pi)=0$.

    En ambos casos, tenemos que $\beta=1,2,3,\dots$, por lo
    cual $\mu=\beta^{2}=n^{2}$ con $n=1,2,3,\dots$.
    
\end{itemize}

Ahora resolvemos para $R(r)$ de la ecuación
\[
  r^{2}R'' +rR' +(r^{2}\lambda^{2}-\mu)R = 0
\]
donde $\mu=0$ o $\mu=n^{2}$ para $n=1,2,3,\dots$.

\begin{itemize}
  \item Si $\lambda=0$, la ecuación es de Cauchy-Euler,
    \[
      r^{2}R''+rR'-n^{2}R = 0
    .\]
    Sustituyendo $R=r^m$ obtenemos soluciones $m=\pm n$, por lo
    cual obtenemos la solución
    \[
      R(r) = K_1r^n+K_2r^{-n}
    .\]
    La condición inicial $R(0)=0$ se convierte en $\lim_{r\to
    0}R(r)=0$, lo cual solo se satisface si $K_2=0$.
    Por otro lado, la condición $R(a)=0$ nos da que $0=K_1a^n$,
    lo cual solo se satisface con $K_1=0$.
    Luego, la solución es trivial.

  \item Si $\lambda\neq 0$ y $\mu=n^{2}$ con
    $n=0,1,2,3\dots$, entonces la ecuación es de
    Bessel de orden $n$:
    \[
      r^{2}R'' +rR' +((\lambda r)^{2}-n^{2})R = 0
    .\]
    Entonces la solución es
    \[
      R(r)
      =
      K_1J_n(\lambda r)
      +
      K_2Y_n(\lambda r)
    ,\]
    donde $J_n$ y $Y_n$ son las $n$-ésimas funciones de
    Bessel de 1er y segundo tipo (o especie).
    \begin{align*}
      J_n(\lambda r)
      &= \sum_{m=0}^{\infty}\frac{(-1)^m}{m!\Gamma(m+n+1)}
      \left(
        \frac{\lambda x}{2}
      \right)^{2m+n}
    \end{align*}

    Las $Y_n(\lambda r)$ no son acotadas cuando $r\to 0$,
    así que la condición $\lim_{r\to r}R(r)=0$ nos hace
    elegir $K_2=0$, por lo cual
    \[
      R(r)=K_1J_n(\lambda r)
    .\]
    Por otro lado, la condición $R(a)=0$ nos da
    $K_1J_n(\lambda a)=0$.
    Así, $\lambda$ son las soluciones de $J_n(\lambda a)=0$.
\end{itemize}

\end{document}
